\documentclass[12pt]{article}
\usepackage[english]{babel}
\usepackage[utf8]{inputenc}
\usepackage{color}
\usepackage[a4paper,lmargin={2cm},rmargin={2cm}, tmargin={2cm},bmargin = {2cm}]{geometry}
\usepackage{amssymb}
\usepackage{amsmath}
\usepackage{mathtools}
\usepackage{graphicx}
\usepackage{cite}
\usepackage{wallpaper}
\usepackage{amstext}
\usepackage{blindtext}
\usepackage{multicol}
\usepackage[singlespacing]{setspace}
\usepackage{tabulary}
\usepackage{tabularx}
\usepackage[T1]{fontenc}
\usepackage{float}
\usepackage{csquotes}

\usepackage{placeins}
\usepackage{afterpage}

\renewcommand{\labelitemi}{$\bullet$}
\renewcommand{\labelitemii}{$\cdot$}
\renewcommand{\labelitemiii}{$\circ$}
\renewcommand{\labelitemiv}{$\ast$}

\usepackage{caption}
\usepackage{framed}
\definecolor{shadecolor}{rgb}{0.6,0.6,0.61}

%\bibliographystyle{alphadin}
\bibliographystyle{IEEEtran}

\begin{document}
%\hyphenation{Ver-schlüsselungs-verfahren}

\begin{titlepage}
\ThisCenterWallPaper{1.0}{h-logo-background}	

%  \HoleMask

%  \hspace{-3cm}
  \begin{minipage}[t]{10cm}
  %\vspace{2cm}
  \includegraphics[width=10cm]{h-logo-full-font-embed}\\
  \end{minipage}
  \vspace{2.5cm}

	\begin{center}

    \vspace{0.8cm}

    \vspace{3cm}
    \begin{Huge}
    \textbf{Exposé}\end{Huge}\\
    \vspace{0.8cm}
  	\begin{huge}
  	\textbf{TITEL}
  	\end{huge}
  	 \vspace{0.6cm} \\
  	  	  \begin{large}von
  	  \end{large}
  	  \\ \begin{LARGE}
  	   \vspace{0.6cm}
  	  {VORNAME NACHNAME}\\
  	  STUDI-ID\\
  	  \vspace{0.8cm}
  	 
  	\end{LARGE}
  	
    \vspace{3.0cm}
			\end{center}
	\begin{large}
%	\begin{flalign}
%	\hspace{4.5cm} 
	%\text{Referee and Tutor:} \hspace{1.2cm}		&\text{Prof. Dr. Alexander Asteroth}& % \notag\\&31.12.2016  \notag
%\end{flalign} 

\begin{table}[h!]
\begin{tabularx}{\textwidth}{l@{\hspace{2.0cm}}X}

Betreut durch: & Prof. Dr. ???\\
und: &  Prof. Dr. ???\\



\end{tabularx}
\end{table}  
  
\end{large}
%  \vspace{0.5cm}
%  \vfill
\end{titlepage}


\pagenumbering{arabic}
\tableofcontents
\newpage{}


\section{Einleitung}

Evolutionäre Taktiken haben sich in der Vergangenheit als mächtige Heuristik zur Untersuchung komplexer Problemräume bewiesen.
Allerdings haben viele evolutionäre Algorithmen die Tendenz, das gefundene Lösungen im Verlauf der Algorithmen immer mehr zu einer Lösungklasse konvergieren.
Das Ergebnis sind damit am Ende n minimal unterschiedliche Variationen der selben Lösung.
Dies führt dazu, dass solche Algorithmen häufig in lokalen Optima stecken bleiben, wenn kein Mechanismus existiert, der nicht-optimale Lösungen, die als Zwischenschritt zu optimaleren Lösungen benötigt werden, zulässt.
Es wurden verschiedene Ansätze vorgeschlagen um die Diversität einer Population zu gewährleisten.
Diversitäts-Ansätze lassen sich allgemein in zwei Kategorien einteilen, solche die Diversität auf Genotyp-Ebene gewährleiste, und solche die diese auf Phänotyp-Ebene gewährleisten.

\section{Problembeschreibung}
 
\section{Literatur}

\section{Ansatz}


\section{Leistungen}
Mindestergebnis:
\begin{itemize}  
\item 
\end{itemize}
Erwartetes Ergebnis:
\begin{itemize}  
\item 
\end{itemize}
Maximalergebnis:
\begin{itemize}  
\item 
\end{itemize}

\newpage{}

%\section{Literature}%

\begin{appendix}
\section{Struktur Thesis}

\section{Zeitplanung und Aufgaben}

\end{appendix}
  
\newpage{}
\section{Literaturverzeichnis}
\bibliography{expose}
\newpage{}

\end{document}}