\documentclass[12pt]{article}
\usepackage[english]{babel}
\usepackage[utf8]{inputenc}
\usepackage{color}
\usepackage[a4paper,lmargin={2cm},rmargin={2cm}, tmargin={2cm},bmargin = {2cm}]{geometry}
\usepackage{amssymb}
\usepackage{amsmath}
\usepackage{mathtools}
\usepackage{graphicx}
\usepackage{cite}
\usepackage{wallpaper}
\usepackage{amstext}
\usepackage{blindtext}
\usepackage{multicol}
\usepackage[singlespacing]{setspace}
\usepackage{tabulary}
\usepackage{tabularx}
\usepackage[T1]{fontenc}
\usepackage{float}
\usepackage{csquotes}
\usepackage{todonotes}
\usepackage{gensymb}

\usepackage{placeins}
\usepackage{afterpage}

\renewcommand{\labelitemi}{$\bullet$}
\renewcommand{\labelitemii}{$\cdot$}
\renewcommand{\labelitemiii}{$\circ$}
\renewcommand{\labelitemiv}{$\ast$}

\usepackage{caption}
\usepackage{framed}
\definecolor{shadecolor}{rgb}{0.6,0.6,0.61}

%\bibliographystyle{alphadin}
\bibliographystyle{IEEEtran}

\begin{document}
%\hyphenation{Ver-schlüsselungs-verfahren}

\begin{titlepage}
\ThisCenterWallPaper{1.0}{h-logo-background}	

%  \HoleMask

%  \hspace{-3cm}
  \begin{minipage}[t]{10cm}
  %\vspace{2cm}
  \includegraphics[width=10cm]{h-logo-full-font-embed}\\
  \end{minipage}
  \vspace{2.5cm}

	\begin{center}

    \vspace{0.8cm}

    \vspace{3cm}
    \begin{Huge}
    \textbf{Exposé}\end{Huge}\\
    \vspace{0.8cm}
  	\begin{huge}
  	\textbf{Optimierung eines Velomobilradkastens sowie einer Abschlussplatte eines E-Rollers durch Surrogat-assistierte Illumination}
  	\end{huge}
  	 \vspace{0.6cm} \\
  	  	  \begin{large}von
  	  \end{large}
  	  \\ \begin{LARGE}
  	   \vspace{0.6cm}
  	  {Jan Kruska}\\
  	  9028385\\
  	  \vspace{0.8cm}
  	 
  	\end{LARGE}
  	
    \vspace{3.0cm}
			\end{center}
	\begin{large}
%	\begin{flalign}
%	\hspace{4.5cm} 
	%\text{Referee and Tutor:} \hspace{1.2cm}		&\text{Prof. Dr. Alexander Asteroth}& % \notag\\&31.12.2016  \notag
%\end{flalign} 

\begin{table}[h!]
\begin{tabularx}{\textwidth}{l@{\hspace{2.0cm}}X}

Betreut durch: & Prof. Dr. Alexander Asteroth\\
und: &  Alexander Hagg, M.Sc.\\



\end{tabularx}
\end{table}  
  
\end{large}
%  \vspace{0.5cm}
%  \vfill
\end{titlepage}


\pagenumbering{arabic}
\tableofcontents
\newpage{}


\section{Einleitung}

Die Domäne der Aerodynamik stellt durch ihre Komplexität sowohl für Mensch als auch Computer eine große Herausforderung dar.
Die physikalischen Gesetze denen diese folgt sind zwar bekannt, deren Zusammenspiel allerdings so stark ineinander verwoben, das beim Design aerodynamischer Körper oft auf Faustregeln zurückgefallen werden muss.
%Durch computergestützte Assistenz könnte dieser Designprozess erheblich verbessert werden.

Evolutionäre Taktiken haben sich in der Vergangenheit als mächtige Heuristik zur Optimierung komplexer Problemräume bewiesen \cite{Vikhar.2016}.
%Die Untersuchung inwieweit solche Taktiken für die Nutzung in der computergestützten Assistenz nützlich sind liegt nahe.
%Eine Methode die sowohl in der Lage ist neben Lösungen auch Wissen über den Lösungsraum zu generieren könnte den Designprozess erheblich vereinfachen indem sowohl Initiallösungen, von denen eine weitere menschliche %Optimierung ausgehen kann, als auch Wissen über das Problem, womit diese weitere Optimierung zielstrebiger erfolgen kann, erzeugt werden.

Genetische Algorithmen sind eine effektive  Methode um kreative Lösungen für komplexe Probleme zu generieren.
Allerdings haben evolutionäre Algorithmen ohne Zusatzmechanismen die Tendenz, dass gefundene Lösungen im Verlauf der Algorithmen zu einer Lösungsklasse konvergieren \cite{Shir.2012}.
Dies führt zu zwei Hauptproblemen. 
Erstens neigen solche Algorithmen dadurch dazu in lokalen Optima "stecken" zu bleiben, da das Verlassen eines lokalen Optimums immer mit einer Fitnesseinbuße verbunden ist.
Zweitens produzieren solche Algorithmen nur eine Lösung. 
Sollten mehrere gute Lösungen existieren, deren Fitness sich nur gering unterscheidet wird trotzdem die Lösung die um einen Bruchteil besser ist bestehen bleiben.
Häufig kann es allerdings sinnvoll sein auch diese Lösungen zu erhalten, da ein Menschlicher Beobachter dadurch mehr Wissen erhalten kann welche Eigenschaften, für gute Lösungen sorgen, wenn sie beispielsweise häufig vorkommen, oder ob bestimmte Eigenschaften nur in Kombination gute Lösungen ergeben.
Dieses Wissen, was aus der Vielzahl an guten Lösungen gezogen werden kann, kann dabei helfen das Problem besser zu verstehen, was eine sehr wertvolle Eigenschaft ist.
%Damit kann der menschliche Beobachter auch nicht einschätzen, ob die erhaltene Lösung tatsächlich die global beste Lösung ist, und wenn sie das nicht ist wie in Richtung des globalen Optimums weiteroptimiert werden kann.

Es wurden verschiedene Diversitätsansätze vorgeschlagen, die sich grundsätzlich in genotypische und phänotypische Diversitätsansätze unterscheiden lassen.
Ansätze wie Niching, gewährleisten genotypische Diversität durch die Ermittlung der genetischen Ähnlichkeit zweier Individuen.
Das hat den Vorteil, dass der Genotyp eines Individuums typischerweise ein Vektor ist, und eine Vielzahl von Ähnlichkeitsmetriken für Vektoren existiert.
Andere Ansätze verfolgen das Ziel phänotypischer, oder funktionaler, Diversität.
Dies hat den Grund, dass die Funktion eines Individuums und nicht die Encodierung das eigentliche Problem ist, und in komplexen Problemdomänen sich genetisch sehr verschiedene Individuen trotzdem funktional ähneln können.
Um phänotypische Diversität zu gewährleisten wird allerdings eine phänotypische Ähnlichkeitsmetrik benötigt, die sehr domänenspezifisch ist.
Diese Ansätze werden allgemein unter dem Begriff der Quality-Diversity-Algorithmen zusammengefasst.
Diese liefern typischerweise eine Vielzahl möglicher Lösungen, durch die das Problem und Zusammenhänge zwischen Lösungseigenschaften und Lösungsqualität besser verstanden werden kann.



\section{Problembeschreibung}

\subsection{Radkastendomäne}
\missingfigure{Radkästen}

Die erste Problemdomäne befasst sich mit der Verformung der Radästen eines Velomobils. 
In Abb. \ref{fig:wheelcase} ist das das Velomobil mit eingezeichneten Radkästen zu sehen.
Es sollen beide Radkästen des Velomobils verformt werden, der Einfachheit halber kann diese Verformung aufgrund der Symmetrie des Velomobils symmetrisch zur xz-Ebene
\footnote{\label{foot:coords} Unter Annahme, dass die x-Achse die horizontale Vor/Rückachse, die y-Achse die horizontale Links/Rechtsachse und die z-Achse die Vertikalachse ist} stattfinden, sodass eine Deformation die Anweisungen für beide Radkästen entält.
Die momentanen Radkästen schränken den vollen Lenkausschlag der Räder des Velomobils ein.
Es ist das Ziel Lösungen zu ermitteln, die den vollen Lenkausschlag ermöglichen.
                      
\subsection{E-Roller-Domäne}
\missingfigure{E-Roller-Abdeckung}

Die zweite Problemdomäne ist ein E-Roller an dessen Unterseite ein Bauteil vor dem Hinterrad verformt werden soll.
In Abb. \ref{fig:escooter} ist der E-Roller zu sehen. Das zu verformende Bauteil ist rot hervorgehoben.
Auch für den E-Roller kann die Verformung symmetrisch zur xz-Ebene \footnote{Siehe Fußnote \ref{foot:coords}} erfolgen um den Suchraum zu verkleinern.
Auch für den E-Roller müssen Constraints eingehalten werden. 
So darf das Bauteil nicht beliebig weit nach unten verformt werden, da dies zu Kollisionen mit der Fahrbahn führen würde.
Gleichzeitig darf das verformte Bauteil bestehende Elemente des E-Rollers nicht schneiden.
Anders als in der Radkastendomäne erfüllt das Basisbauteil diese Constraints schon, Ziel ist also nur die Verbesserung des Luftwiederstands ohne dabei die Constraints zu verletzen.
 
\section{Literatur}

\subsection{MAP-Elites}
\label{sub:mapElites}
MAP-Elites \cite{Mouret.4202015} ist ein genetischer Algorithmus der phänotypische Diversität gewährleistet.
Dies geschieht durch eine Aufteilung des Lösungsraum entlang n Dimensionen, die Eigenschaften der Lösung darstellen.
Dadurch wird der Lösungsraum in phänotypische Zellen eingeteilt.
Jede dieser Zellen kann eine Lösung enthalten, die nur durch eine andere bessere Lösung, die in die gleiche Zelle passt ersetzt werden kann.
Neben einer Vielzahl von Lösungen kann diese Karte auch dabei helfen die Problemdomäne besser zu verstehen, indem Zusammenhänge zwischen den Kategorien, nach denen klassifiziert wurde und der Qualität der Lösungen, aufgedeckt werden.
\todo[inline]{umformulieren, umstrukturieren, Quelle}

\subsection{Surrogat-Modell}
\label{sub:surrogate}
Für die Selektion von Individuen innerhalb eines genetischen Algorithmus wird die Lösungsqualität dieser Individuen, typischerweise Fitness genannt, benötigt.
Außerdem findet die Auswertung der Fitnessfunktion innerhalb des genetischen Algorithmus sehr häufig statt.
Dies stellt bei relativ einfachen Fitnessfunktionen keine große Einschränkung dar, auf Problemdomänen in denen die Auswertung der Fitness eines Individuums allerdings komplexer und dadurch zeitaufwändiger wird, kann dies die Anwendbarkeit einfacher genetischer Algorithmen einschränken.

Aerodynamische Probleme, für die zeitaufwendige Simulationen nötig sind, gehören ohne Zweifel zu der Klasse von Problemen für die die Anzahl der benötigten Funktionsauswertungen zu groß sind, als das der Algorithmus in vertretbarer Laufzeit abschließen kann.
Um genetische Taktiken auf eine solche Problemdomäne anzuwenden, wird eine Möglichkeit benötigt die benötigten Funktionsauswertungen erheblich zu reduzieren.
Eine solche Möglichkeit ist ein Surrogatmodell \cite{Jin.2011}\cite{Preen.2016}, eine Machine-Learning Modell, welches aufgrund echter Simulationsauswertungen trainiert wird um deren Ergebnis annähernd vorherzusagen.
Eine Auswertung des Modells erfordert dabei nur einen winzigen Bruchteil des Aufwands der für eine Simulation nötig wäre.
Theoretisch sind verschiedenste Machine-Learning Verfahren für das Surrogatmodell denkbar, Gaußprozesse bieten sich durch die Eigenschaft an, dass sie neben einer Vorhersage auch immer ihre eigene Unsicherheit Liefern, und entsprechend Bereiche zeigen können in denen große Unsicherheit herrscht, sprich in denen der Gaußprozess wenig weiß.
Der Nachteil der hohen Komplexität von Gaußprozessen \todo[inline]{genaue Komplexität? Quelle: \cite{Rasmussen.2008}} ist durch die Limitierung auf eine geringe Anzahl an Simulationen, und damit der Limitierung auf einen kleinen Trainingsdatensatz vernachlässigbar.
\todo[inline]{Abschnitt erfolgreiche Anwendung von Surrogat-Modellen und genetischen Algorithmen, sth. about Orders of magnitude evaluation reduction}

\subsection{SAIL}

SAIL (Surrogate-Assisted Illumination) vereint die in \ref{sub:mapElites} und \ref{sub:surrogate} beschriebenen Ansätze.
In \cite{Gaier.6152018} wurde gezeigt, dass dieser Ansatz erfolgreich auf 2D und 3D aerodynamische Domänen angewandt werden kann.
\todo[inline]{Inwieweit sollen Anpassungen vorgenommen werden? Feature-Dimensionen,zusätzliche Contraints etc. }

\todo[inline]{Thesis von Sascha zwar nicht zitierfähig aber draufeingehen doch vermutlich sinnvoll}
\section{Ansatz}

\subsection{Velomobil}
Aufgrund der Tatsache, dass im momentanen Zustand der Constraint noch nicht erfüllt ist ist ein harter Constraint, der solche Lösungen, die ihn nicht erfüllen nicht zulässt nur schwer umzusetzen, da am Anfang des Prozesses keine bekannte valide Lösung existiert.
Weiche Constraints lassen es zu, dass Lösungen existieren die den Constraint erfüllen, in der Hoffnung, dass diese als Trittbrett zu Lösungen, die ihn erfüllen genutzt werden können.
In einer vorigen Arbeit wurde ein solcher Ansatz verfolgt, allerdings wurde hier ein konstanter Strafwert addiert, wenn der Constraint verletzt war.
Man könnte einen variablen Strafwert einfügen, der abhängig davon wie stark der Constraint verletzt wird größer oder kleiner ist, um dem Algorithmus einen Gradienten zu validen Lösungen zu geben.
\todo[inline]{Wie formuliert man diesen variablen Strafwert, muss schnell berechnbar sein}

Als Fehlermaß für die Qualität der am Ende produzierten Lösungen kann der Anteil am maximalen Radausschlag der erreicht wurde genutzt werden.
\todo[inline]{konkrete formel, schnitt kugel radkasten? oder schnitt kreise in  bspw. 1$\degree$ azimuth}

Um Verformungen an den Ränder der Radkästen, und die dadurch resultierenden scharfen Kanten zu vermeiden, sollten primär innere Punkte als Deformationspunkte gewählt werden.

Als Kategorien für MAP-Elites sind die Gesamtbreite des Velomobils interessant, da die Hypothese aufgestellt werden kann, dass geringe Breiten den Contraint nicht erfüllen, größere Breiten allerdings die Aerodynamik verschlechtern.
Daneben sind die Höhe des Velomobils, das Volumen und die Krümmung des Radkastens interessant.
Sowohl die Berechnung des Volumens als auch die der Krümmung sind dabei nicht-trivial und aufgrund der Häufigkeit dieser müssten vermutlich Annäherungen herangezogen werden.
Es kann aber auch die Hypothese aufgestellt werden, dass das Volumen als auch Krümmung positiv mit der Breite des Velomobils korreliert sind.
Daneben sind allerdings auch noch andere Kategorien denkbar, falls sich im Laufe der Experimente Zusammenhänge aufzeigen sollten, die genauerer Untersuchung bedürfen.

Die Überprüfung der Hypothese der Existenz eines Breitenoptimums, außerhalb dessen Lösungen entweder den Constraint nicht mehr erfüllen, oder sich die aerodynamischen Eigenschaftenen verschlechtern stellt eine interessante Forschungsfrage dar.
Falls Kategorien gewählt werden für die die Annahme besteht, dass diese korreliert sind, könnte auch diese Annahme durch Experimente überprüft werden.

\subsection{E-Roller}
Anders als bei Velomobil ist die Nutzung von Hard-Contraints beim E-Roller einfacher möglich, um das Einführen von invaliden Lösungen zu verhindern.
Um die Komplexität des Problems zu vermindern kann man sich auf Verformungen in z-Richtung beschränken.
Dadurch können die Freiheitsgrade auf mehr Deformationspunkte aufgeteilt werden.

Interessante Kategorien für den E-Roller könnten Höhe, Krümmung und Volumen des Bauteils sein.
Die Höhe ist trivial aus den Verformungsparametern berechenbar.
Für Krümmung und Volumen müssen Annäherungen, die weniger rechenaufwendig, sind genutzt werden.
Vor allem für das Volumen könnten allerdings aufgrund der Tatsache, dass eine rechteckige Grundfläche in eine Richtung verformt wird schnelle und gute Annäherungen existieren.
Auch hier könnten bei Bedarf andere Kategorien herangezogen werden.

Genau wie bei den Radkästen kann die Hypothese einer positiven Korrelation zwischen Höhe und Volumen, bzw. Krümmung aufgestellt werden.
Die Validierung oder Widerlegung dieser Hypothese könnte Einsicht in das Problem des Designs dieses Bauteils schaffen


\section{Leistungen}
Mindestergebnis:
\begin{itemize}  
\item Die Implementierung des produziert für die Radkastendomäne eine Karte aus Verformungen der Radkästen.
\end{itemize}
Erwartetes Ergebnis:
\begin{itemize}  
\item  Die Implementierung liefert für die Radkastendomäne Lösungen die 90\% des maximalen Radausschlags ermöglichen.
\todo[inline]{Vllt. noch? Hypothese der Korrelation zwischen Kategorien im Radkastenfall konnte bestätigt oder widerlegt werden}
\end{itemize}
Maximalergebnis:
\begin{itemize}  
\item Die Implementierung liefert für die E-Roller-Domäne eine Karte aus Lösungen
\item Die Implementierung liefert für die E-Roller-Domäne Lösungen die die Constraints erfüllen
\item Eine Auswahl der besten gefunden Lösungen wurden auf dem nicht-vereinfachten E-Roller-Mesh verifiziert
\todo[inline]{Vllt. noch? Hypothese der Korrelation zwischen Kategorien im E-Rollerfall konnte bestätigt oder widerlegt werden}
\end{itemize}

\newpage{}

%\section{Literature}%

\begin{appendix}
\section{Struktur Thesis}

\begin{enumerate}
	\item Eidesstattliche Erklärung
	\item Abstract
	\item Grundlagen
	\begin{itemize}
		\item Genetische Algorithmen
		\item MAP-Elites
		\item Surrogat-Modell
		\item Probleme \& Herausforderungen von bestehenden Ansätzen \todo{Sascha?}
	\end{itemize}
\item Methode
\begin{itemize}
	\item Radkastendomäne
	\item E-Roller-Domäne
	\item Wahl der Features
	\item Design-Constraints und das daraus resultierende Design der Fitnessfunktion
\end{itemize}
\item Ergebnis Radkastendomäne
\begin{itemize}
	\item Experimente
	\item Analyse
	\item Zusammenfassung
\end{itemize}
\item Ergebnis E-Rollerdomäne
\begin{itemize}
	\item Experimente
	\item Analyse
	\item Zusammenfassung
\end{itemize}
\item Diskussion
\item Ausblick
\item Literaturverzeichnis
\item Anhang
\end{enumerate}

\section{Zeitplanung und Aufgaben}

\end{appendix}
  
\newpage{}
\section{Literaturverzeichnis}
\bibliography{citavi}
\newpage{}

\end{document}}