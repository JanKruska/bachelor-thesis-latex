%###############################################################################
% Template fuer eine Abschlussarbeit mit LaTeX

\NeedsTeXFormat{LaTeX2e}
\documentclass[a4paper]{book}

% LaTeX-Zusatzpakete
%\usepackage{german}                % deutsches Dokument (Trennung etc.)
\usepackage[latin1]{inputenc}      % deutsche Eingabezeichen (�� etc. erlaubt)
\usepackage{times}                 % Times Font
\usepackage{graphicx}              % Grafiken einbinden (h�ngt von latex/dvipdf oder pdflatex ab!)
\usepackage{tabularx}              % erweiterte Tabellenm�glichkeiten
\usepackage{subfig}                % Teilgrafiken erlauben
\usepackage{amsfonts}              % fuer zus�tzliche math. Symbole
\usepackage{amssymb}               % weitere math. Symbole
\usepackage{theorem}               % Theorem-Umgebung
\usepackage{todonotes}
%\usepackage{listings}              % Programmlistings
% Steuerung fuer Programmlistings (anpassen an Programmiersprache)
%\lstset{language=Java, frame=single, breaklines=true, tabsize=2}


%###############################################################################
% Gr�ssenangaben zum Dokument etc.

% vertikal
\setlength{\voffset}{-0.5cm}
\setlength{\textheight}{23cm}
\setlength{\topmargin}{0cm}
\setlength{\headheight}{6mm}
\setlength{\headsep}{1cm}
\setlength{\topskip}{0cm}
\setlength{\footskip}{1cm}

% horizontal
\setlength{\hoffset}{-0.4cm}
\setlength{\textwidth}{15.5cm}
\setlength{\oddsidemargin}{0.8cm}
\setlength{\evensidemargin}{0.8cm}

\setlength{\parindent}{0cm}        % kein Einzug bei Paragrafenbeginn


%###############################################################################
% Hier geht es los

% Autor und Abgabedatum �ndern
\def\autor{Jan Kruska}
\def\datum{\date}

% Titelseite erzeugen
\begin{document}


%###############################################################################
% Titelseite

\sloppy
\pagestyle{headings}
\bibliographystyle{alphadin}
\pagenumbering{roman}


% Basierend auf der Fachschaftsvorlage
\begin{titlepage}
  \textbf{
    \begin{tabular}[t]{>{\vspace*{0pt}}p{0.45\textwidth}c>{\vspace*{0pt}}p{0.45\textwidth}}
      {
        \begin{tabular}{ll}
          \begin{tabular}{l}
            \raisebox{1.5\height}{\includegraphics[width=0.13\textwidth]{fhlogo}} 
          \end{tabular}
          &
          \begin{tabular}{l}
            {\small{\sffamily Hochschule}}\\
            {\small{\sffamily Bonn-Rhein-Sieg}}\\
            {\footnotesize{\itshape University of Applied Sciences}}\\
            \\
              {\small{\sffamily Fachbereich Informatik}}\\
              {\footnotesize{\itshape Computer Science Department}}\\
          \end{tabular}
        \end{tabular}
      }
      &
      \hspace*{0.1\textwidth}
      &
      \begin{tabular}{r}
% hier evtl. Firma nennen mit/ohne Logo, ansonsten auskommentieren
%        \hfill \includegraphics[width=0.2\textwidth]{firmenlogo}\\
      \end{tabular}
    \end{tabular}
  } %\textbf
  
  
  \renewcommand{\baselinestretch}{1.4}\normalsize
  
  \vspace{3cm}
  \begin{center}
    
% einen Typ ausw�hlen
    \begin{Huge}\textbf{Abschlussarbeit}\end{Huge}\\
    \vspace{0.8cm}
% einen Studiengang ausw�hlen
    \begin{Large}\textbf{im Bachelorstudiengang Informatik}\end{Large}\\
    
    \vspace{2.2cm}
    \renewcommand{\baselinestretch}{1.2}\normalsize
    \begin{huge}
      \textbf{Thema der Arbeit\\}
    \end{huge}
    \renewcommand{\baselinestretch}{1.5}\normalsize
    \vspace{0.7cm}
    
    
    \begin{Large}\textbf{von \autor\ \\}
    \end{Large}
  \end{center}
  
  \vspace{5.0cm}
  
  \begin{large}
    \textbf{
      \begin{tabular}{ll}
      Erstbetreuer:  & Prof. Dr.-Ing Franz Mustermann\\
      Zweitbetreuer: & Dipl.-Ing Kloria Musterfrau\\
                     & \\
      Eingereicht am: & \datum\ \\
      \end{tabular}
    }
  \end{large}
\end{titlepage}


%###############################################################################
% Erkl�rung

\cleardoublepage
\section*{Erkl�rung}

Hiermit erkl�re ich an Eides Statt, dass ich die vorliegende Arbeit
selbstst�ndig und ohne Benutzung anderer als der angegebenen
Hilfsmittel angefertigt habe. Die aus fremden Quellen direkt oder
indirekt �bernommenen Gedanken sind als solche kenntlich gemacht.  Die
Arbeit wurde bisher in gleicher oder �hnlicher Form noch keiner
anderen Pr�fungsbeh�rde vorgelegt bzw. nicht ver�ffentlicht.

\vspace{1cm}
Sankt Augustin, den \datum\
\vspace{2cm}


\rule{10cm}{0.1mm} \\
\autor\


%###############################################################################
% Abstract

\clearpage
\section*{Abstract}
  Englische Zusammenfassung

\clearpage
\section*{Zusammenfassung}
  Diese Arbeit befasst sich mit grunds�tzlichen �berlegungen zur
  Lastverteilung in Parallelrechnern mit einer gro�en
  Prozessoranzahl. Als wesentliches Ergebnis dieser Arbeit wird
  nachgewiesen, dass die Skalierbarkeit eines Algorithmus
  grunds�tzlich nach oben durch die Anzahl der Prozessoren bechr�nkt
  ist.


%###############################################################################
% Inhaltsverzeichnis

\tableofcontents           % Inhaltsverzeichnis generieren
\cleardoublepage
\pagenumbering{arabic}


%###############################################################################
% Kapitel 1

\chapter{Einleitung}
ca. 1-2 Seiten

Motivation f�r die Arbeit

Aufgabenbeschreibung, Abgrenzung


%###############################################################################

\chapter{Grundlagen}

Wie schon in \cite{Goldberg1991} \cite{Balzert2001} \cite{Rump1988}
gesagt wurde, ist das alles sehr wichtig. Auch \cite{Gosling2005} und
\cite{Java2005} haben dies gezeigt.


\section{Unterkapitel}

\subsection{Unterunterkapitel}

\subsubsection{Unterunterunterkapitel}
Eine zu tiefe Gliederung des Textes erh�ht nicht die Lesbarkeit.


%###############################################################################

\chapter{Konzept}

Bilder sind manchmal gut zur Darstellung von komplexen
Zusammenh�ngen. Dazu dient hier die Abbildung \ref{fig:Gliederung}.

\begin{figure}
\begin{center}
\includegraphics[height=2cm]{fhlogo}
\caption{�bersichtsbild.}
\label{fig:Gliederung}
\end{center}
\end{figure}


%###############################################################################

\chapter{Realisierung}

Zwei Bilder sind nicht immer besser als ein Bild, wie Abbildung
\ref{fig:ZweiBilder} zeigt.

\begin{figure}
\centering
\subfloat[Teilbild 1]{\includegraphics[width=0.2\hsize]{fhlogo}}
\hspace*{0.05\hsize}
%\subfloat[Teilbild 2]{\includegraphics[width=0.2\hsize]{firmenlogo}}
\caption{Zwei Bilder in einem Bild.}
\label{fig:ZweiBilder}
\end{figure}


%###############################################################################

\chapter{Evaluierung}


%###############################################################################

\chapter{Zusammenfassung und Ausblick}


%###############################################################################
% Literaturverzeichnis und ggfs. Index

\cleardoublepage
\addcontentsline{toc}{chapter}{Literaturverzeichnis}
\bibliography{literatur}

% soweit gew�nscht: Tabellen- und Abbildungsverzeichnis
\cleardoublepage
\listoftables              % Tabellenverzeichnis
\cleardoublepage
\listoffigures             % Abbildungsverzeichnis


%###############################################################################
% Appendix (falls n�tig)

\cleardoublepage
\addcontentsline{toc}{chapter}{Anhang}
\begin{appendix}           % Anhang

\chapter{Liste aller Kommandos}
Im Anhang finden sich technische Details, die f�r den eigentlichen
Text zu umfangreich sind.  Es ist aber auch zu fragen, ob solche
umfangreichen Informationen in einer schriftlichen Ausarbeitung
�berhaupt sinnvoll sind und nicht besser elektronisch zur Verf�gung
gestellt werden.

\chapter{Ablaufprotokoll}
...

\end{appendix}


%###############################################################################

\end{document}

%###############################################################################
