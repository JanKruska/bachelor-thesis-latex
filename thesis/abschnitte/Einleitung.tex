
\clearpage
\section*{Abstract}
Evolutionary Algorithms have proved to be a powerful optimization heuristic, to both examine and solve complex problems.
Through approaches such as surrogate-modelling, they are applicable even in computationally expensive domains.
A common problem to arise however, are constraints.
Constraints are secondary aim that need to be either fulfilled or optimized alongside the primary goal, depending on the type of constraint.
Constraints often occur as a result of real world restrictions placed upon solutions generated by an evolutionary algorithm.
As such it is often hard to correctly quantify constraints.
Additionally constraint need to be integrated into the evolutionary algorithm, without interfering in its core functionality.
This thesis tries to answer the question how such a constraint can be integrated into a divergent evolutionary algorithm and if the integration of this constraint influences the diversity of such an algorithm, one of the main advantages of divergent evolutionary algorithms.
The algorithm at hand is the surrogate-assisted Illumination (SAIL) algorithm, which extends the existing divergent evolutionary algorithm MAP-Elites with a surrogate modelling approach, to make it applicable in computationally expensive domains.

\clearpage
\section*{Zusammenfassung}

%Genetische Algorithmen sind ein mächtiges Werkzeug zur Untersuchung komplexer Problemräume. Allerdings stellt die hohe Anzahl an benötigten Funktionsauswertungen ein Problem in solchen Domänen dar in denen die Funktionsauswertung mit erheblichem Rechenaufwand verbunden ist. Durch Surrogat-assistierte Ansätze können die benötigten Funktionsauswertungen um mehrere Größenordnungen reduziert werden, was ermöglicht genetische Algorithmen auch auf solche Problemdomänen anzuwenden. Trotzdem fehlen häufig die Ressourcen für eine multivariate Optimierung, wenn sekundäre Optimierungsziele existieren müssen diese als Constraints in Verfahren eingebunden werden. Die Arbeit untersucht in zwei Problemdomänen, wie Constraints innerhalb des Rahmens der Surrogat-assistierten Illumination, einer phänotypisch divergenten Optimierungsmethode, die MAP-Elites um Surrogat-Assistenz erweitert, verwirklicht werden können.
Evolutionäre Algorithmen sind ein bewährtes Optimierungsverfahren, zur Untersuchung komplexer Probleme.
Durch Verfahren wie der Surrogatassistenz können diese auch in Domänen, in denen die Berechnung der Fitness rechenintensiv ist, weil diese beispielsweise durch eine Simulation berechnet wird, genutzt werden.
Häufig tritt bei der praktischen Anwendung von evolutionären Algorithmen das Problem auf, dass neben der Zielfunktion weitere Constraints existieren, welche optimiert beziehungsweise erfüllt sein sollten.
Bei der Modellierung von Constraints stellen sich einige Fragen:
Wie sollen diese mathematisch formuliert werden? 
Wie sollen diese in den evolutionären Algorithmus integriert werden?
Und kann trotz des Constraints eine diverse Population generiert werden
In dieser Arbeit soll untersucht werden, ob und wie ein Constraint in einen divergenten evolutionären Algorithmus integriert werden kann und welchen Effekt der Constraint auf die Eigenschaften des evolutionären Algorithmus hat.
Dies soll am Beispiel der Surrogat-assistierten Illumination (SAIL), einem auf MAP-Elites basierten divergenten evolutionären Algorithmus mit Surrogatassistenz, geschehen.

\section{Einleitung}

Evolutionäre Algorithmen haben sich in der Vergangenheit als mächtige Heuristik zur Untersuchung komplexer Problemräume bewiesen.\todo{Quelle}
Evolutionäre Algorithmen sind dabei in der Lage kreative Lösungen, in komplexesten Problemräumen zu ermitteln.
Genetische Algorithmen sind eine solche Taktik, die an der biologischen Evolution angelehnt entwickelt wurde.
Genetische Algorithmen entwickeln erst eine Möglichkeit eine funktionale Lösung, auch Phänotyp genannt, durch einen Genotypen, typischerweise ein Vektor, ausdrücken zu können.
Mit einer Mapping-Funktion kann aus jedem Genotypen eine konkrete Lösung, ein Phänotyp, erstellt werden.
Außerdem wird eine Zielfunktion oder Fitnessfunktion benötigt, mit der die Lösungsqualität eines Individuums, d.~h. dessen Phänotyps, ermittelt werden kann.
Sind diese Dinge definiert, können die drei Mechanismen der Evolution, Mutation, Crossover und Selektion angewandt werden.
Bei der Anwendung evolutionärer Algorithmen stellen sich allerdings einige Probleme.

Eines dieser Probleme stellt sich recht häufig, da genetische Algorithmen ohne Diversitätsmanagement typischerweise sehr stark konvergieren und dadurch oft in lokalen Optima hängen bleiben.
Um dieser Tendenz entgegenzuwirken und diverse Populationen aufrechtzuerhalten wurden verschiedene Taktiken vorgeschlagen, die sich grob in genotypische und phänotypische Diversitätsmanagementverfahren unterscheiden lassen.
Verfahren zur Erhaltung genotypischer Diversität messen Diversität, die nichts anderes als das Inverse der Ähnlichkeit einer Population untereinander ist, durch die Ähnlichkeit zweier Genotypen zueinander.
Da Genotypen Vektoren sind, kann hier auf eines der vielen Ähnlichkeitsmaße, die für Vektoren existieren, zurückgegriffen werden.
Daneben existieren auch solche Verfahren, die Diversität anhand der Ähnlichkeit der Phänotypen messen.
Das hat den Vorteil, das der Phänotyp die Funktion, eines Individuums beschreibt, es wird also funktionale , statt genetischer Diversität hergestellt.
Allerdings bringt dies das weitere Problem mit sich wie man die phänotypische Ähnlichkeit zweier Individuen definiert.
Da der Phänotyp stark von der Problemdomäne abhängt, muss diese Ähnlichkeitsdefinition für jede Domäne unterschiedlich erfolgen.
Im Folgenden soll sich mit einer Möglichkeit zum phänotypischen Diversitätsmanagement auseinandergesetzt werden.

Daneben ergeben sich häufig domänenspezifische Probleme.
Eines dieser Probleme ist, dass genetische Algorithmen eine große Zahl von Auswertungen der Fitnessfunktion benötigen.
Das kann dazu führen, dass genetische Algorithmen in Problemdomänen, in denen die Auswertung der Fitnessfunktion rechenaufwändig ist, in einer unveränderten Form,
aufgrund der astronomisch hohen Rechenzeiten, die benötigt würden, schlicht nicht anwendbar sind.
Um genetische Algorithmen für solche Probleme nutzen zu können werden somit Möglichkeiten benötigt, die Anzahl an rechenaufwändigen Evaluationen stark zu reduzieren.
Eine solche Möglichkeit, die Erfolg gezeigt hat, sind Surrogat-Modelle.
Diese werden mit tatsächlichen Evaluationen trainiert, deren Ergebnisse hervorzusagen.
Das führt dazu, dass sie statt der tatsächlichen Evaluation im genetischen Algorithmus genutzt werden können.
Die Auswertung eines Surrogat-Modells benötigt dabei nur einen Bruchteil der Zeit, die eine tatsächliche Evaluation benötigen würde.

Das letzte große Problem, was sich stellt, ist die Einbindung von Constraints.
Häufig existieren neben der primären Zielfunktion die optimiert wird, weitere sekundäre Ziele.
Wenn die Ressourcen für eine multivariate Optimierung fehlen, können solche sekundären Optimierungsparameter nicht einfach zu Zielfunktionen erklärt werden.
Stattdessen müssen diese Constraints anderweitig in die Algorithmik eingebunden werden.
Es gibt verschiedene Ansätze zu Integration von Constraints in Optimierungsprobleme, welcher dieser Ansätze aber am besten funktioniert hängt stark von der Problemdomäne und den Constraints ab.

In dieser Arbeit soll untersucht werden wie sich verschiedene Arten Constraints in die Algorithmik eines divergenten genetischen Verfahren auf die, von diesem Verfahren erzeugten Lösungen, auswirkt.
Dies soll anhand der Lösung eines Problems der Aerodynamik-Domäne durch Anwendung der Surrogat-assistierten Illumination geschehen, einem Verfahren welches den divergenten genetischen Algorithmus MAP-Elites um eine Surrogat-Assistenz erweitert.
Insbesondere die Frage, ob eine Einbindung von Constraints in einen solchen divergenten evolutionären Algorithmus so möglich ist, dass sowohl die eigentliche Zielfunktion, in diesem Falle die aerodynamischen Eigenschaften, als auch der Constraint optimiert werden.
Es ist zu erwarten, dass durch die Einbindung eines Constraints die Optimierung der Zielfunktion schlechter ausfallen wird.
Es gilt diesen Qualitätsverlust zu quantifizieren und in Relation zum Gewinn bezüglich der Erfüllung des Constraints zu setzen.
Auch die Frage ob und wenn ja wie stark, ein solcher Constraint sich auf die Diversität der Lösungen auswirkt, gilt es zu untersuchen.

