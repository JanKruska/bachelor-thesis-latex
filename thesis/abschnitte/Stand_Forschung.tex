\section{Grundlagen}

\subsection{Genetische Algorithmen}

Genetische Algorithmen sind eine Klasse von Optimierungsalgorithmen, die sich an den Mechanismen der natürlichen Evolution orientieren.
Namentlich Mutation, Crossover und Selektion.
Die Begrifflichkeiten orientieren sich entsprechen an der natürlichen Evolution.
Genetische Algorithmen arbeiten typischerweise mit Populationen aus Individuen, welche sich über Generationen entwickeln.
Jedes Individuum besitzt einen Genotypen sowie einen Phänotypen.
\todo{Bild geno phäno}
Der Genotyp eines Individuums ist typischerweise ein Vektor, das Genom des Individuums genannt, der aus einzelnen Zahlen, den Genen des Individuums, besteht. Dieser stellt die genetischen Informationen eines Individuums dar. Daneben existiert in genetischen Algorithmen eine Mapping-Funktion mit der eine Genotyp in einen Phänotyp, der nichts anderes als eine konkrete Lösung des Optimierungsproblems ist, übersetzt werden kann.

Die Mechanismen der Mutation und des Crossovers agieren auf der Ebene des Genotyps.
Mutation beschreibt die Situation, dass eine gewisse Wahrscheinlichkeit existiert, mit der sich ein Gen ändern kann. Abhängig vom Typen des Gens, ist eine Änderung unterschiedlich formuliert, eine reelle Zahl mag sich anhand einer Normalverteilten Zufallsvariable ändern, eine natürliche Zahl mit $\pm1$ und ein Index mag einen zufälligen möglichen Index annehmen. Wichtig ist nur, dass potenziell neue Eigenschaften in eine Population eingeführt werden können.
Crossover beschreibt die Situation, dass zwei oder mehr Elternindividuuen zu einem Kindindividuum kombiniert werden. Auch hier ist die Implementierung nicht vorgeschrieben, wichtig ist nur, dass ein Kind als Kreuzung der Eltern erzeugt werden kann.
Es ist allerdings anzumerken, dass Crossover einen optionalen Teil von genetischen Algorithmen darstellt, einige Algorithmen, wie beispielsweise ($1 + \lambda$)-ES verzichten aus verschiedenen Gründen auf Crossover und mutieren ihre Populationen nur.
Diese beiden Methoden stellen sicher, dass neue Genotypen, und damit neue Lösungen generiert werden können, stellen aber keine Mechanismus zur Verfügung durch den der Algorithmus optimale Lösungen erreicht.
Der Mechanismus, der der Optimierung eine Richtung gibt ist die Selektion.
Diese agiert auf Phänotypen, sprich ausgedrückten Lösungen.
Für jede Lösung kann die Güte oder der Kosten einer Lösung, entsprechend Fitness oder Kosten genannt, 
mit einer Fitness- oder Kostenfunktion berechnet werden.
Ob eine Fitness- oder Kostenfunktion genutzt wird hängt von der Problemformulierung ab, und ist letztendlich eine Frage ob minimiert oder maximiert wird.
Die Fitness, bzw. der Kosten jedes Individuums kann dann ermittelt werden und die Selektion eliminiert dann schlechte Lösungen, sprich solche mit niedriger Fitness, bzw. hohen Kosten.
Dadurch werden in jeder Generation neue Individuen generiert und schlechte Individuen aussortiert, was Stück für Stück zu einer Verbesserung der Population führt.

\subsubsection{MAP-Elites}

\subsection{Surrogat-Modelle}

\subsection{Gaußprozesse}

Gaußprozesse sind ein in \todo{cite gp} entwickeltes Machine-Learning Verfahren, mit welchem beliebige mathematische Funktionenen approximiert werden können. in großer Vorteil von Gaußprozessen ist, dass sie durch ihre Herkunft aus der Statistik neben die zu approximierende Funktion als Kombination aus Normalverteilungen modellieren.
Dadurch können Gaußprozesse neben einer Vorhersage auch die Varianz der Vorhersage an jedem Punkt liefern.
Das bedeutet, dass zu jedem Punkt auch die Unsicherheit bekannt ist, was für die Exploration des Funktionsraum
ein erheblicher Vorteil ist.
Der Hauptnachteil von Gaußprozessen ist der relativ hohe Speicher und Rechenkomplexität, da die Anzahl an Funktionsauswertungen 
\subsection{Constraints in Optimierungsprozessen}
Constraints sind eine Möglichkeit sekundäre Optimierungsparemter, welche neben dem primären Optimierungsziel ebenfalls eingehalten werden sollen, in einen Optimierungsprozess einzuarbeiten, ohne auf multivariate Optimierung zurückgreifen zu müssen.
Constraints lassen sich grundsätzlich in Soft Constraints, bei denen die Nichteinhaltung des Constraints zur Addition von Strafwerten auf die Optimierungsfunktion führt, und Hard Constraints, die eine binäre erfüllt/nicht erfüllt Auswahl treffen.
Welche dieser beiden Arten von Constraints genutzt wird, hängt stark von der Problemstellung ab.

\todo{Formulierung von Constraints aus nicht-mathematischer Formulierung}
\subsubsection{Hard Constraints}
Hard Constraints führen eine binäre Auswahl durch, bei der solche Lösungen, die den Constraint nicht erfüllen disqualifziert werden.
Sie stellen eine sehr einfache Lösung zur Implementierung von Constraints dar.
An der Stelle im Algorithmus an der neue Lösungen generiert werden werden alle ungültigen Lösungen herausgefiltert.
Dies hat den Vorteil, dass ungültige Lösungen niemals in den Algorithmus einfließen, und damit weder Rechenkapazitäten für nicht-nutzbare Lösungen genutzt werden, und solche Eigenschaften, durch die eine Lösung die Constraints verletzt vom Algorithmus überhaupt nicht in Erwägung gezogen werden.


\subsubsection{Soft Constraints}
Soft Constraints disqualifizieren Lösungen, die die Constraints verletzen, nicht.
Stattdessen wird beim Nichterfüllen von Constraints ein Strafwert auf die Kostenfunktion addiert, bzw. von einer Fitnessfunktion subtrahiert.
Das Lösungen, die die Constraints nicht erfüllen, nicht disqualifiziert werden, hat den Vorteil, dass viele Optimierungmethoden iterativ zu (lokalen) Optima konvergieren\footnote{Auch ein divergentes Optimierungsverfahren wie MAP-Elites konvergiert zu Optima, es wird nur sichergestellt, das zu einer Vielzahl lokaler Optima konvergiert wird}, 
und diese auch Lösungen, die die Constraints nicht erfüllen, als Trittbretter zu Lösungen, die die Constraints erfüllen, nutzen können.

Soft Constraint eignen sich in Fällen, in denen zu Beginn keine Lösungen bekannt sind, die die Constraints erfüllen, und in denen explorativ nach Lösungen gesucht werden soll, die die Constraints erfüllen.
Auch eignen sie sich für solche Probleme, in denen qualitative Unterschiede bezüglich der Stärke der Verletzung der Constraints zwischen unterschiedlichen Lösungen exisistieren können.
So kann argumentiert werden, dass im Falle, dass ein Constraint die Einhaltung eines Schwellenwerts ist, eine Lösung, die diesen um 1 überschreitet, qualitativ besser ist als eine, die diesen um 10 überschreitet.

Eine der größten Gefahren bei Soft Constraints ist, dass diese typischerweise als Kostenfunktionen formuliert werden, deren Wert mit der eigentlichen Zielfunktion der Optimierung kombiniert wird.
Eine solche Kombination enthält immer eine Gewichtung für alle Teilfunktionen die sie ausmacht.
Ist diese Gewichtung fehlerhaft parametrisiert, kann dies dazu führen, dass Teilfunktionen über- oder unterpriorisiert werden, und Constraints vom Algorithmus ignoriert werden, oder, dass nicht mehr nach dem primären Optimierungsziel optimiert wird.




\section{Methode}

\subsection{Radkästen des Velomobils}

Die erste Problemdomäne stellt die Optimierung der Radkästen eines Velomobils dar.
Da die momentanen Radkästen den Lenkausschlag des Velomobils erheblich einschränken, besteht das Ziel hierbei Radkästen zu generieren, die den maximalen oder zumindest einen weiteren Lenkausschlag ermöglichen und dabei trotzdem gute aerodynamische Eigenschaften aufweisen.

\subsubsection{Constraint}
Zur Erfüllung des Constraints wurden alle möglichen Radausschläge als Volumen modelliert (Volumen zu finden in ffd/turning\_volume.stl)

\begin{table}[h]
	\begin{tabularx}{.5\textwidth}{ll}\hline
		
		Position\footnote{Urpsprung des Koordinatensystems auf Boden an der Spitze des Velomobils. Linkshändiges Koordinatensystem mit z Höhenachse und Velomobil in -x-Richtung ausgerichtet} & \\
		\hline
		x &	$926mm$ \\
		y &	$343mm$ \\
		z &	$205mm$	\\
	\end{tabularx}
	\begin{tabularx}{.5\textwidth}{ll}\hline
		Rotation & \\ \hline
		$\phi$ & $16,6\degree$ \\
		$\theta$ & $0\degree$ \\
		$\psi$ & $0\degree$ \\
	\end{tabularx}
	\begin{tabularx}{.5\linewidth}{ll}
		Radius des Rads & $230mm$ \\
		
		Radausschlag & $\pm24,37\degree$\\
		
	\end{tabularx}

\label{tab:wheel_params}
\caption{Parameter Radausschlag}
\end{table}

Da ein Ziel darin bestand, den Constraint nicht als binäres, erfüllt/nicht erfüllt Problem zu definieren, da zwischen zwei Lösungen, die den Constraint nicht erfüllen trotzdem qualitative Unterschiede bestehen können wie stark der Constraint verletzt wird, wird als Constraint das Differenzvolumen des Radausschlags minus verformten Radkastens gewählt.
Da alle generierten Verformungen der Radkästen symmetrisch sind, wird der Constraint jeweils nur für den rechten Radkasten berechnet.
Zwar besteht keine direkte Kausalität zwischen diesem Volumen und dem maximalen möglichen Radausschlag aber die Vermutung, dass eine geringeres Volumen dieser Different mit größerem möglichen Radausschlag korreliert ist liegt nahe.
Zur Berechnung der Differenz zwischen Radausschlagsvolumen und Radkasten wird die Bibliothek \textit{gptoolbox} \todo{cite} genutzt, zur Generierung von Tetraedermeshes zur Volumenberechnung \textit{TetGen} \todo{cite}.
Um eine Differenz berechnen zu können wurde der rechte Radkasten zu einem geschlossenen Volumen gemacht, da die Constraintberechnung in jeder Generation für jedes Kind erfolgen muss wurde der Radkasten außerdem runtergesamplet \todo{klingt komisch? besseres wort} um die Berechnung  des Constraints ausreichend schnell durchführen zu können.
Aus dem Volumen wurde dann der Strafwert berechnet der in die Akquisefunktion integriert wurde.

\subsection{Kategorien}
Als Kategorien wurde die Breite des Velomobils gewählt.
Hierzu kann die Hypothese aufgestellt werden, dass die Breite des Velomobils mit der Erfüllung des Constraint korreliert ist. Es wäre also zu erwarten, dass breitere Velomobile den Constraint tendenziell besser erfüllen, als schmalere.
Auch kann die Hypothese aufgestellt werden, dass die Breite negativ mit dem Luftwiderstandsbeiwert korreliert ist.
Insgesamt wäre also zu vermuten, dass entlang dieser Achse
Als zweite Kategorie wurde die x-Koordinate des breitesten Punkts gewählt.
Da das Velombil in negative x-Richtung zeigt, bedeuten kleinere Werte hier, dass der Punkt weiter vorne liegt, größere, dass er weiter hinten liegt.
Zu dieser Kategorie lassen sich keine so direkten Hypothesen aufstellen wie zur ersteren.
Die Frage ob sich hier klare Tendenzen bezüglich den aerodynamischen Eigenschaften und/oder des Constraints aufzeigen ist Ziel der Untersuchung.


\subsubsection{E-Roller}

Die zweite Problemdomäne ist die explorative Untersuchung eines Bauteils an der Unterseite eines E-Rollers.
Hier besteht die Frage ob nicht-triviale Bauteile aerodynamische Vorteile bieten, und welche Eigenschaften solche Bauteile aufweisen.

\subsection{OpenFOAM}

OpenFoam \todo{cite openfoam} ist eine Open-Source Programm zur Durchführung von Fluiddynamiksimulationen.
Alle Dateien für OpenFOAM-Simulationen sind in den Ordnern \textit{domains/wheelcase/pe} und \textit{domains/escooter/pe} zu finden.
\todo{irgendetwas zu cases}
OpenFOAM wird durch Dictionary-Dateien gesteuert.
Mit diesen Dictionary-Dateien werden die in OpenFOAM enthaltenen Funktionen parametrisiert.
OpenFOAM bietet native Unterstützung von Parallelität über die MPI-Bibliothek \todo{cite mpi}.
Neben den Funktionen um die korrekte Aufteilung eines Cases auf mehrere Prozessoren aufzuteile und am ENde wieder zu rekonstruieren sind zwei Funktionen und deren dazugehörige Dateien hervorzuheben.

Die erste dieser Funktionen ist \textit{snappyHexMesh}, das durch \textit{snappyHexMeshDict} parametrisiert wird, mit welchem das aus dem STL-Orginalmesh das für die Fluiddynamiksimulation benötigte Mesh generiert wird. Durch die Parametrisierung von snappyHexMesh wird kontrolliert wie detailliert das Mesh an welchen Stellen ist.
Die wichtigen genutzten Definitionen in snappyHexMesh sind:
\todo{snappyhexmesh definitionen}

Die zweite dieser Funktione ist die eigentliche Fluiddynamiksimulation welche durch die Datei \textit{fvSolution} gesteuert wird. 
Es stehen verschiedene Simulationen für verschiedene Zwecke zur Verfügung.
Hier wurde \textit{simpleFoam} gewählt, da der Funktionsumfang für den Fokus dieser Arbeit völlig ausreicht.


