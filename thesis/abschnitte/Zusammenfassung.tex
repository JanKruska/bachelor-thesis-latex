\section{Diskussion}


\section{Ausblick}

\subsection{Radkasten}

\subsection{E-Roller}

Aufgrund der Schwierigkeiten der Anwendung von SAIL auf die E-Rollerdomäne lag der Fokus der Untersuchung dieser Arbeit in der Radkastendomäne.
Das Hauptproblem der in dieser Arbeit erzeugten Ergebnisse ist die Unsicherheit bezüglich deren Korrektheit.
Der Grund dafür ist der Fakt, dass die cD-Werte in der gewählten Konfiguration OpenFoam nicht konvergierten.
Der Grund dafür kann eine fehlerhafte Parametrisierung Openfoams sein, die Anpassung dieser Konfiguration sprengt allerdings den Umfang dieser Arbeit.
In zukünftigen Untersuchungen sollte also, vor der Nutzung von SAIL, zuerst sichergestellt werden, dass die zugrundeliegende Simulation in Openfoam realitätsgetreue Ergebnisse liefert.
Ist das nicht garantiert können alle möglichen Ergebnisse von SAIL nur beschränkte Aussagekraft haben.

Des weiteren sind die erzeugten Formen für das Bauteil nicht besonders kreativ oder bahnbrechend.
Das kann daran liegen, dass für dieses Bauteil keine guten unintuitiven Lösungen existieren.
Die Anzahl an möglichen Lösungen wird allerdings durch die gewählte Deformationskonfiguration eingeschränkt.
Es sollte möglicherweise über andere FFD-Konfigurationen, beziehungsweise Deformatiormationen in anderen Punkten und/oder Richtungen nachgedacht werden.
Besonders, dass die seitlichen Ränder des Bauteils in der hier gewählten Konfiguration nicht deformiert werden können stellt eine starke Einschränkung dar.
Auch sollte in Erwägung gezogen werden die linke und rechte Hälfte symmetrisch zu verformen.
Erstens ist anzunehmen, dass dadurch, dass der E-Roller an sich symmetrisch ist die Ströumunge die das Bauteil treffen ebenso symmetrisch sind.
Lösungen die in solchen Strömungen am besten abschneiden werden mit hoher Wahrscheinlichkeit symmetrisch sein.
Im Experiment konnte auch beobachtet werden, dass alle erzeugten Bauteile fast symmetrisch waren.
Alle Asymmetrien waren meist leicht und eher auf die Größe des genetischen Suchraums zurückzuführen, als auf einen Vorteil durch eine asymmetrische Bauart.
Auch ist fragwürdig ob  asymmetrische Bauteile überhaupt in die Designparameter des E-Rollers passen.
Ästhetik ist schwer quantifizierbar und Fertigungstauglichkeit sowie Einbettbarkeit des Bauteils in den E-Roller wurden in dieser Arbeit nicht untersucht, alle drei könnte aber von vornherein gegen asymmetrische Bauteile sprechen.
Durch die symmetrische Verformung des Bauteils könnte allerdings die Anzahl an benötigten Freiheitsgraden halbiert, beziehungsweise mit der gleichen Anzahl an Freiheitsgraden feinere Deformationen erlaubt werden.
In Anbetracht dieser drei Tatsachen scheinen symmetrische Verformungen bei denen die Verformungen der linken und rechten Hälfte des Bauteils aus dem gleichen Genom abgeleitet werden sinnvoller.

In dem durchgeführten Experiment waren die besten Ergebnisse bezüglich des Luftwiderstands solche die effektiv flach oder leicht nach oben deformiert waren.
Es besteht die Möglichkeit, dass nach oben deformierte Bauteile noch optimaler sind.
Dieser Lösungsraum sollte nach Lösungen untersucht werden.
Allzu stark deformiert Lösungen wurden in  dem durchgeführten Experiment durch die beschränkte minimale Deformation ausgeschlossen.
Es wäre möglich mehr Deformationen nach oben zu erlauben um solche Lösungen genauer zu untersuchen.
Zwar muss dann eine Möglichkeit gefunden werden, dass solche Bauteile, den restlichen E-Roller nicht schneiden, sodass beispielsweise um dessen Bauteile herum deformiert wird.
Dies könnte beispielsweise durch die Einführung eines Constraints geschehen, der testet, ob das Bauteil mit anderen Teilen des E-Rollers kollidiert.

Durch die Wahl des tiefsten Punkt als eines der Features war es selbst für leicht nach oben deformierte Bauteile schwer Einzug in die Karte zu erhalten.
Deshalb sollte zusätlich über die die Kategorien zu nachgedacht werden, nach denen die Bauteile phänotypisch kategorisiert werden.
Die gewählten Kategorien wurden hauptsächlich aus dem Grund gewählt, dass diese einfach berechenbar und klar verständlich sind.
Es wäre vermutlich sinnvoll für die Domäne abgewägte Kategorien zu wählen, und zu untersuchen wie sich dies auf die Lösungen auswirkt.

Zuletzt stellte die benötigte Laufzeit für OpenFoam trotz der Nutzung des vereinfachten E-Roller Modells eine nicht unbeträchliche Hürde dar.
Dies ist vor allem auf die komplexere Form des E-Rollers im Vergleich zur aerodynamisch optimierten Projektilform des Velomobils und die längere Simulationsdauer von 1000 Sekunden zurückzuführen.
Eine qualitative Untersuchung welche Teile des E-Rollers wegrationalisiert werden können, ohne signifikante Ungenauigkeit in die Ergebnisse einzuführen, könnte die Laufzeit reduzieren wodurch beispielsweise wieder präzise Funktionsauswertungen ermöglicht werden, was mehr Freiheitsgrade erlauben kann.