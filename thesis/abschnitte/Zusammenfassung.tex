\section{Diskussion}


\section{Ausblick}

\subsection{Radkasten}

\subsection{E-Roller}

Aufgrund der Schwierigkeiten der Anwendung von SAIL auf die E-Rollerdomäne lag der Fokus der Untersuchung dieser Arbeit in der Radkastendomäne.
Das Hauptproblem der in dieser Arbeit erzeugten Ergebnisse ist die Unsicherheit bezüglich deren Korrektheit.
Der Grund dafür ist der Fakt, dass die cD-Werte in der gewählten Konfiguration OpenFoam nicht konvergierten.
Der Grund dafür kann eine fehlerhafte Parametrisierung Openfoams sein, die Anpassung dieser Konfiguration sprengt allerdings den Umfang dieser Arbeit.
In zukünftigen Untersuchungen sollte also, vor der Nutzung von SAIL, zuerst sichergestellt werden, dass die zugrundeliegende Simulation in Openfoam realitätsgetreue Ergebnisse liefert.
Ist das nicht garantiert können alle möglichen Ergebnisse von SAIL nur beschränkte Aussagekraft haben.

Des weiteren sind die erzeugten Formen für das Bauteil nicht besonders kreativ oder bahnbrechend.
Das kann daran liegen, dass für dieses Bauteil keine guten unintuitiven Lösungen existieren.
Die Anzahl an möglichen Lösungen wird allerdings durch die gewählte Deformationskonfiguration eingeschränkt.
Es sollte möglicherweise über andere FFD-Konfigurationen, beziehungsweise Deformatiormationen in anderen Punkten und/oder Richtungen nachgedacht werden.
Besonders, dass die seitlichen Ränder des Bauteils in der hier gewählten Konfiguration nicht deformiert werden können stellt eine starke Einschränkung dar.

Zuletzt sind noch die Kategorien zu erwähnen nach denen die Bauteile phänotypisch kategorisiert worden.
Die gewählten Kategorien wurden hauptsächlich aus dem Grund gewählt, dass diese einfach berechenbar und klar verständlich sind.
Es wäre vermutlich sinnvoll für die Domäne abgewägte Kategorien zu wählen, und zu untersuchen wie sich dies auf die Lösungen auswirkt.

Zuletzt stellte die benötigte Laufzeit für OpenFoam trotz der Nutzung des vereinfachten E-Roller Modells eine nicht unbeträchliche Hürde dar.
Dies ist vor allem auf die komplexere Form des E-Rollers im Vergleich zur aerodynamisch optimierten Projektilform des Velomobils und die längere Simulationsdauer von 1000 Sekunden zurückzuführen.
Eine qualitative Untersuchung welche Teile des E-Rollers wegrationalisiert werden können, ohne signifikante Ungenauigkeit in die Ergebnisse einzuführen, könnte die Laufzeit reduzieren wodurch beispielsweise wieder präzise Funktionsauswertungen ermöglicht werden, was mehr Freiheitsgrade erlauben kann.